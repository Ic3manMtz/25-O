%%%%%%   TIPO DE DOCUMENTO: Artículo   %%%%%%
\documentclass[letterpaper,11pt,spanish]{report}

% ========================================
% CONFIGURACIÓN BÁSICA DEL DOCUMENTO
% ========================================
\usepackage[utf8]{inputenc}        % Soporte para caracteres especiales y acentos
\usepackage[spanish]{babel}        % Idioma español (traducciones y reglas tipográficas)
\usepackage{xspace}                 % Manejo inteligente de espacios después de comandos
\renewcommand{\baselinestretch}{1} % Interlineado simple

% --------------------------
% CONFIGURACIÓN DE TÍTULOS 
% --------------------------

%\titleformat{comando}{forma}
%   {formato}
%   {etiqueta}
%   {separacion}
%   {antes del titulo}
%   [despues del titulo]

\usepackage{titlesec}
\usepackage{etoolbox}

\titleformat{\chapter}[display] 
{\Large\raggedleft} 
{}
{0pt}
{
    \ifnum\value{chapter}>0 
        \rule{\textwidth}{0.5pt}
        \vspace{1ex}
        {\Large\bfseries\chaptername\ \thechapter \\}
        \vspace{1ex}
    \fi
} 
[
    \rule{\textwidth}{0.3pt}
] 


\titleformat{\section}
{\normalfont\bfseries\Large\raggedright}
{\thesection}
{1em}
{}

\titleformat{\subsection}
{\normalfont\bfseries\large\raggedright}
{\thesubsection}
{1em}
{}

% --------------------------
% CONFIGURACIÓN DE ENCABEZADOS CON FANCYHDR
% --------------------------
\usepackage{fancyhdr}

\renewcommand{\chaptermark}[1]{\markboth{#1}{}}
\fancypagestyle{mainstyle}{%
  \fancyhf{}
  \fancyhead[L]{\leftmark}
  \fancyhead[R]{\thepage}
  \fancyfoot{}
}

% --------------------------
% CONFIGURACIÓN DE ANEXOS
% --------------------------
\newcommand{\appendixtitleformat}{
  \titleformat{\chapter}[display]
    {\Large\raggedleft}
    {\appendixname\ \thechapter}
    {1ex}
    {\Large\bfseries}
    [\vspace{1ex}\rule{\textwidth}{0.3pt}]
}


% ========================================
% MATEMÁTICAS Y SÍMBOLOS
% ========================================
\usepackage{amsmath}                % Entornos matemáticos avanzados
\usepackage{amssymb}                % Símbolos matemáticos extendidos
\usepackage{amscd}                  % Diagramas conmutativos
\usepackage{amsthm}                 % Teoremas y entornos de demostración

% ========================================
% ALGORITMOS Y PSEUDOCÓDIGO
% ========================================
\usepackage{algorithm}              % Entornos para algoritmos
\usepackage{algpseudocode}          % Estilo para pseudocódigo

% ========================================
% GRÁFICOS E IMÁGENES
% ========================================
\usepackage{graphicx}               % Inclusión y manipulación de imágenes
\usepackage{subcaption}             % Subtítulos para subfiguras (subfigure)
\usepackage{tikz}                   % Creación de gráficos vectoriales
\usetikzlibrary{shapes, arrows}     % Formas y flechas para TikZ
\usetikzlibrary{arrows.meta}        % Estilos avanzados de flechas
\usetikzlibrary{positioning}        % Posicionamiento preciso de nodos

% ========================================
% CÓDIGO FUENTE Y LISTADOS
% ========================================
\usepackage{listings}               % Inclusión de código fuente
\renewcommand{\lstlistingname}{Código} % Cambia "Listing" por "Código"
\renewcommand{\lstlistlistingname}{Lista de Códigos} % Título de la lista de códigos
\lstset{                            % Configuración básica de listados:
    basicstyle=\ttfamily\small,     
    numbers=left,
    frame=shadowbox,
    breaklines=true,               
    captionpos=b,                   
}

% ========================================
% ESTRUCTURAS DE DIRECTORIOS
% ========================================
\usepackage{dirtree}                % Generación de árboles de directorios

% ========================================
% LISTAS Y ENUMERACIONES
% ========================================
\usepackage{enumitem}               % Personalización de listas (espaciados, etiquetas)
\setlist{itemsep=0pt, topsep=5pt}   % Configuración de espacios en listas

% ========================================
% TABLAS Y ELEMENTOS FLOTANTES
% ========================================
\usepackage{booktabs}               % Tablas con mejor formato
\usepackage{float}                  % Control avanzado de posición de flotantes
\usepackage{caption}                % Personalización de leyendas
\usepackage{calc}                   % Cálculos precisos de dimensiones

% ========================================
% HIPERVÍNCULOS Y REFERENCIAS
% ========================================
\usepackage{hyperref}               % Hipervínculos en el documento
\usepackage[active]{srcltx}         % Búsqueda inversa en editores (doble clic en PDF)

% ========================================
% OTROS UTILITARIOS
% ========================================
\usepackage{verbatim}               % Entornos para texto sin interpretación
\usepackage{makeidx}                % Generación de índices

% ========================================
% CONFIGURACIONES ADICIONALES
% ========================================
\renewcommand{\thesection}{\thechapter.\arabic{section}} % Numeración arábiga para secciones
\renewcommand{\thesubsection}{\thechapter.\arabic{section}.\arabic{subsection}}

% --------------------------
% CONFIGURACIÓN DE PÁGINA
% --------------------------
\setlength{\textheight}{21.6cm}
\setlength{\textwidth}{14cm}
\setlength{\oddsidemargin}{1cm}
\setlength{\evensidemargin}{1cm}
\captionsetup[figure]{position=below, skip=0pt}

% --------------------------
% INICIO DEL DOCUMENTO
% --------------------------
\begin{document}

% --------------------------
% PORTADA
% --------------------------
\thispagestyle{empty}

\begin{figure}[h]
    \centering
    \includegraphics[scale=0.6]{img/logoUAM.png}
\end{figure}

\begin{center}
    \rule{\textwidth}{0.5pt}
    \\[0.7em]
    {\LARGE\bfseries Pr\'actica 2: Exploratory Data Analysis}\\[1cm]
    {\normalsize\itshape Presentado por:}\\
    {\large Jorge Rafael Martínez Buenrostro}
    \rule{\textwidth}{0.5pt}
\end{center}

\vspace{2cm}
\begin{center}
    {\large Profesor: Ren\'e Mac Kinney Romero}    
\end{center}

\vfill  
\begin{center}
    México, CDMX, a \today
\end{center}

\vspace{2cm}

\pagenumbering{roman}
\setcounter{page}{0}
\pagestyle{plain}
% --------------------------
% RESUMEN (ABSTRACT)
% --------------------------
%\chapter*{Resumen}
%Una trayectoria se define como la secuencia de desplazamientos realizada por un individuo en movimiento, compuesta por tres elementos principales: \texttt{puntos de recorrido}, \texttt{tiempos de pausa} y \texttt%%{longitudes de vuelo}. Los puntos de recorrido corresponden a las ubicaciones o pasos específicos por los que transita el individuo. Los tiempos de pausa representan los intervalos durante los cuales el individuo permanece detenenido en un mismo punto de recorrido. Por último, la longitud de vuelo se refiere a la distancia recorrida entre dos puntos de recorrido consecutivos. En el presente trabajo se describe el proceso de caraterización de datos de movilidad, es decir, el proceso de limpieza y depuración de la información, mediante el cual elimina aquellos campos y registros que no le aportan valor a la trayectoria individual. El objetivo es identificar la mayor cantidad de trayectorias individuales, para así poder crear un modelo que permita simular el movimiento de individuos.

% --------------------------
% ÍNDICE DE CONTENIDOS
% --------------------------
\newpage
%\renewcommand{\contentsname}{Contenido}
%\tableofcontents

% --------------------------
% LISTA DE FIGURAS
% --------------------------
%\renewcommand{\listfigurename}{Lista de Figuras}
%\listoffigures

% --------------------------
% LISTA DE CODIGOS FUENTE
% --------------------------
%\cleardoublepage
%\addcontentsline{toc}{chapter}{Lista de Códigos}
%\lstlistoflistings

% --------------------------
% CUERPO PRINCIPAL
% --------------------------
\cleardoublepage
\pagestyle{mainstyle} % Aplica el estilo de encabezado definido
\pagenumbering{arabic}
\setcounter{page}{1} % Reinicia numeración después de índices

% Capítulos 
%\chapter{Introducción del Proyecto}
%\input{Secciones/introduccion}

\cleardoublepage
El análisis exploratorio de datos (EDA) es un paso crucial en el proceso de análisis de datos, ya que permite comprender mejor los conjuntos de datos mediante diversas técnicas. El objetivo principal del EDA es extraer variables significativas, identificar valores atípicos o errores y explorar las relaciones entre las variables, maximizando así los conocimientos y minimizando los posibles errores en futuros análisis.

El EDA cumple dos funciones principales: limpiar el conjunto de datos y mejorar la comprensión de las variables y sus interrelaciones. Este proceso se puede dividir en tres componentes esenciales:

\begin{itemize}
    \item \textbf{Comprensión de la variables:} esto implica comandos como \textsl{.shape}, \texttt{.head()}, \texttt{.columns}, \texttt{.nunique} y \texttt{.decribe()} para obtener información sobre la estructura, los valores y las estadísticas básicas del conjunto de datos. Es fundamental comprender lo que representan los datos antes de sumergirse en el análisis.
    \item \textbf{Limpieza del conjunto de datos:} este paso incluye eliminar variables redundantes, seleccionar columnas pertinentes (eliminando aquellas con un número elevado de valores nulos), eliminar valores atípicos basándose en rangos lógicos y descartar filas con datos faltantes. También puede implicar reclasificar variables discretas para mejorar la claridad.
    \item \textbf{Análisis de las relaciones entre variables:} Este componente utiliza visualizaciones, como matrices de correlación (mapas de calor) para discernir relaciones generales, diagramas de dispersión para investigar relaciones específicas entre dos variables, histogramas para examinar la distribución de variables individuales y diagramas de caja para identificar valores atípicos y obtener información sobre la distribución de los datos. 
\end{itemize}

Estas representaciones visuales facilitan la comunicación rápida de información compleja y son esenciales para seleccionar modelos adecuados y perfeccionar el proceso general de análisis de datos.



\end{document}

